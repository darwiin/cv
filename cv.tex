%!TEX TS-program = xelatex

% Ajouter print entre les crochets pour générer une version monochrome du CV
\documentclass[]{roger-cv}
% \addbibresource{bibliography.bib}

\begin{document}
\header{Christophe}{Roger}
       {architecte logiciel | développeur/concepteur JEE}


% In the aside, each new line forces a line break
\begin{aside}
  \section{about}
    31 rue Smith
    69002 Lyon
    France
    ~
    \href{mailto:adrien@friggeri.net}{adrien@friggeri.net}
    \href{http://friggeri.net}{http://friggeri.net}
    \href{http://facebook.com/adrien}{fb://adrien}
  \section{languages}
    bilingual french/english
    spanish \& italian notions
  \section{programmation}
    {\color{red} $\varheartsuit$} Java (JEE, JSE, JME, JavaCard Platform)
    .NET (C\#, asp.net)
  \section{web}
    {JSP, JSF, JBoss RichFaces, PHP, XML,Javascript, XHTML, LESS, CSS}
  \section{bases de données}
    {IBM DB2, Oracle Database, Microsoft SQL Server, MySQL, PostGreSQL}
  \section{EDIs}
    {Rational Software Architect, Websphere Integration Developer, Visual Studio, Netbeans}
  \section{middleware}
    {Websphere Application Server, Websphere Procss Server, Infosphere Traceability Server, Glassfish, Tomcat}
  \section{conception}
    {UML, Merise}
\end{aside}

\section{expérience}

\begin{entrylist}
  \entry
    {Aujourd'hui}
    {avril 2012}
    {Architecte logiciel}
    {BULL SAS, Sophia Antipolis, France}  
    {
      \emph{IT Specialist} pour IBM, Industry Solutions Insurance\\ 
      \vspace{-3mm}
      \begin{itemize}
        \item{Analyse et simplification de l'architecture existante}
        \item{Mise en place d'un framework de développement d'interface web (jquery, bootstrap, taglibs)}
        \item{Implémentation de nouvelles fonctionnalités du produit}
        \item{Implémentation, analyse et livraison de correctifs de bugs}
       \end{itemize}
    }

  \entry
    {mars 2012}
    {avril 2011}
    {Ingénieur consultant}
    {Altran, Sophia Antipolis, France}  
    {
      \emph{IT Specialist} pour IBM, Industry Solutions Insurance\\ 
      Développement d'une application android dans le cadre d'une solution à destination 
      des clients dans le domaine des assurances.
    }

  \entry
    {février 2011}
    {février 2010}
    {Ingénieur consultant}
    {Altran, Sophia Antipolis, France}  
    {
      \emph{IT Specialist} pour IBM, Product Lifecycle Management Center of Excellence\\
      \vspace{-3mm}
        \begin{itemize}
           \item Intégration d'ENOVIA V6, Oracle E-Business Suite et Maximo Asset Management grâce au middleware IBM (Websphere Process Server).
 	\item Intégration de PTC Windchill, Rational DOORS et IGE+XAO Electrical Expert grâce au middleware IBM (Websphere Process Server)
         \end{itemize}
    }

  \entry
    {janvier 2010}
    {décembre 2007}
    {Ingénieur consultant}
    {Altran, Sophia Antipolis, France}  
    {
      \emph{IT Specialist} pour IBM, Sensor Solutions Center of Excellence\\
      \vspace{-3mm}
          \begin{itemize}
 	    \item Mettre en place le suivi et le contrôle des commandes et approvisionnements à l'aide de la RFID
 	    \item Implémentation d'une solution de suivi et d'authentification de containers pour \href{http://www.container-centralen.com/}{CC} (\href{http://www.container-centralen.co.uk/rfid/history.aspx}{description du projet}, \href{http://www.container-centralen.co.uk/rfid/user\%20guide\%20for\%20scanning.aspx}{guide d'utilisation})
               \item Modification et extension d'une solution existante de contrôle des interventions techniques dans un centre de données 
 	    (Intégration avec Maximo Asset Management for IT, utilisation de la RFID pour contrôler l'installation des serveurs\ldots)
 	    \item Implémentation d'une solution de lutte contre la contrefaçon pour un fabriquant de vins et spiritueux (Utilisation de la RFID)
 	    \item Travail d'étude sur le protocole ONS
 	    \item Maintenance corrective et extension d'une plateforme M2M
 	\end{itemize}
    }

\end{entrylist}

\section{scolarité}

\begin{entrylist}
  \entry
    {2007}
    {}
    {Master STIC Professionel {\normalfont filière MBDS}}
    {Université de Nice Sophia Antipolis, France}
    {Master Informatique filière Multimédia, Base de Données et intégration de Systèmes}
  \entry
    {2004}
    {}
    {BTS Informatique de Gestion}
    {Lycée du Grand Nouméa, Dumbéa, Nouvelle-Calédonie}
    {BTS Informatique de Gestion option administrateurs de réseaux}
  \entry
    {2003}
    {}
    {Baccalauréat Scientifique}
    {Lycée Jules Garnier, Nouméa, Nouvelle-Calédonie}
    {Spécialité Mathématiques}
\end{entrylist}

\section{interests}

complex networks, social networks, community detection, community structure,
overlapping communities, information diffusion, viral marketing, social
inference, recommendation, data mining

\section{applications}

\begin{entrylist}
  \entry
    {2012}
    {}
    {Who did I forget ?}
    {\href{http://whodidiforget.com}{whodidiforget.com}}
    {Guest list recommendation for Facebook events based on friends already attending the event.}
  \entry
    {2011}
    {}
    {Fellows}
    {\href{http://fellows-exp.com}{fellows-exp.com}}
    {Automatic community detection among Facebook Friends in order to validate the \emph{cohesion} measure, creation of friend lists.}
  \entry
    {2008}
    {}
    {Happy Flu}
    {\href{http://happyflu.com}{happyflu.com}}
    {Experiment aimed to measure viral spreading of content across the blogosphere.}
\end{entrylist}

\section{publications}


\printbibsection{article}{article in peer-reviewed journal}
\begin{refsection}
  \nocite{*}
  \printbibliography[sorting=chronological, type=inproceedings, title={international peer-reviewed conferences/proceedings}, notkeyword={france}, heading=subbibliography]
\end{refsection}
\begin{refsection}
  \nocite{*}
  \printbibliography[sorting=chronological, type=inproceedings, title={local peer-reviewed conferences/proceedings}, keyword={france}, heading=subbibliography]
\end{refsection}
\printbibsection{misc}{other publications}
\printbibsection{report}{research reports}

\end{document}
